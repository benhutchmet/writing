\documentclass{article}
\usepackage{graphicx} % Required for inserting images

% for setting up figures
\usepackage{subfigure}
\usepackage{multirow}

% package for margin spacing
\usepackage[margin=1.5in,inner=1.5in]{geometry}
\geometry{top=1.0in}

% package for math stuff
\usepackage{algorithmic}
\usepackage{algorithm}
\usepackage{mathtools}
\usepackage{amssymb}
\usepackage{amsmath}


\usepackage[style=science]{biblatex}
\addbibresource{refs.bib}

\title{Meeting 31st March}
\author{Ben Hutchins}
\date{2023}

\begin{document}

\maketitle

\section*{Rossby waves}

\subsection*{Introduction}

Rossby waves are inertial waves which occur in rotating fluids due to the conservation of vorticity. In the atmosphere, these take the form of large-scale meanders in the high level winds which circle the poles. In the ocean they propogate along the thermocline and are caused by wind stress anomalies at the surface.

\subsection*{Physical basis}

Rossby waves occur in an inviscid, barotropic fluid of constant depth, such as the atmosphere.

\begin{itemize}
    \item Inviscid - viscous (frictional) forces are considered to be zero in the free atmosphere.
    \item Barotropic - surfaces of constant pressure and constant density coincide ($\nabla \rho \times \nabla p = 0$). Pressure depends only on density. 
\end{itemize}

In this case, the divergence of the horizontal velocity is zero, given by:

\begin{equation}
    \nabla_h \cdot \mathbf{\vec{u}} = \frac{\partial u}{\partial x} + \frac{\partial v}{\partial y} = 0
    \label{eq:divergence}
\end{equation}

The vorticity of the fluid is given by the curl of the velocity:

\begin{equation}
    \nabla \times \mathbf{\vec{u_a}} = \frac{\partial v}{\partial x} - \frac{\partial u}{\partial y} + f  = {\eta}
    \label{eq:absolute_vorticity}
\end{equation}

The absolute vorticity is given by the curl of the absolute velocity ($\mathbf{\vec{u_a}}$), which is the air velocity relative to an inertial frame. For this we consider only the vertical component of the relative vorticity, $\zeta = \frac{\partial v}{\partial x} - \frac{\partial u}{\partial y}$, and the vorticity of the earth, $f$ to give the absolute vorticity: $\eta = \zeta + f$.\\

For the barotropic, non-divergent (equation \ref{eq:divergence}) fluid case, the absolute vorticity is conserved following the motion:

\begin{equation}
    \frac{D}{Dt}(\zeta + f) = 0
    \label{eq:conservation_of_absolute_vorticity}
\end{equation}

% material derivative? having some issues with understanding this

In the more general case, we consider a baroclinic atmosphere where density depends on both temperature and pressure, where the geostrophic wind varies with height. In this case, the Rossby wave is a potential vorticity-conserving motion which occurs due to gradients in potential vorticity. The potential vorticity is given by:

\begin{equation}
    \mathcal{Q} = \alpha (2 \mathbf{\Omega} + \nabla \times \mathbf{u}) \cdot \nabla \theta
    \label{eq:potential_vorticity}
\end{equation}

Where $\mathbf{\Omega}$ is the angular velocity of the earth, $\alpha$ is the specific volume ($1/\rho$), $\mathbf{u}$ is the three dimensional vector velocity, and $\theta$ is the potential temperature ($\theta = T (p_0/p)^{\kappa}$). In the absence of friction and heating, the potential vorticity (ertel PV, $\mathcal{Q}$) is conserved following the motion of the fluid.\\

\subsubsection*{Emergence of Rossby waves}

For an idealized case, we consider a barotropic fluid with a constant density and depth, where variation in the coriolis parameter is given by:

\begin{equation}
    f = f_0 + \beta y
    \label{eq:coriolis_parameter}
\end{equation}

Where:

\begin{description}
    \item[$f_0$] is the coriolis parameter at $\phi_0$, the latitude of the equator ($y=0$).
    \item[$\beta \equiv (df/dy)_{\phi_0} = 2 \Omega \cos{\phi_0/a}$]  is the Rossby parameter which accounts for the the meridional variation of the Coriolis force caused by the spherical shape of the earth. Where $\Omega$ is the angular velocity of the earth, and $a$ is the radius of the earth.
    \item[$y$] is the distance from the equator.
\end{description}


\subsection*{Relevance}


\section*{Eddy mean flow}

\subsection*{Introduction}


\subsection*{Physical basis}


\subsection*{Relevance}


\section*{PhD update}

% short term goals
% longer term goals
% deliverables - presentations/MC2
% research ideas and questions



% what is the mechanism for this in the oceans
% gain momentum from wind stress at the ocean surface

\printbibliography

\end{document}
